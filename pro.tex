\documentstyle[12pt]{article}
\setlength{\oddsidemargin}{0in}
\setlength{\evensidemargin}{0in}
\setlength{\textwidth}{6.5in}
\setlength{\topmargin}{-.3in}
\setlength{\textheight}{9in}
\pagestyle{empty}
%----------------------------------------------------------------------------------------
%	TITLE SECTION
%----------------------------------------------------------------------------------------

\title{	
\normalfont \Large 
\textsc{Visualization Project: Campus Safety and Security} \\ [10pt] % Your university, school and/or department name(s)
%\horrule{0.5pt} \\[0.4cm] % Thin top horizontal rule
Mohsen Abbasi\footnote{UID: u1011952, Email: m.abbasi@utah.edu} \\ % The assignment title
Amir Biglari\footnote{UID: u0613945, Email: amir.biglari@utah.edu}\\
Waiming Tai\footnote{UID: u1008421, Email: u1008421@utah.edu}
%\horrule{0.5pt} \\[0.4cm] % Thick bottom horizontal rule
}

%\author{Mohsen Abbasi \\ Ashkan Bashardoust} % Your name

%\date{\normalsize\today} % Today's date or a custom date

\begin{document}

\maketitle
\noindent
\textbf{Introduction:}\\
Choosing an institution is a major decision for students and their families. Along with academic, financial and 
geographic considerations, the safety of campus is a vital concern. We're planning to work on "The Campus Safety and Security"\footnote{http://ope.ed.gov/campussafety} data set to create a tool which could be used to address this issue.\\ 

\noindent
\textbf{Goals:}\\
By visualizing this data set, we have a few objectives in mind:
\begin{itemize}
\item Answering basic questions such as: 1. Which states have the safest campuses across the United States? 2. Which schools are facing some specific type of criminal activities more than the others? 3. Which schools are similar considering a particular type of crime?   
\item We're interested in comparing different schools in details and finding out what types of crimes which play a more significant role in threatening the safety of schools.
\item As mentioned above, subjects such as hate crimes and violence against women are included in the data set. It will be interesting to combine this data set with other ones containing demographic information, GDP and such for different states and look for correlations between the crime rates and these attributes. 
\end{itemize}

\noindent
\textbf{Data and Preprocessing:}\\
The Campus Safety and Security data set consists of the statistics for different types of criminal activities at postsecondary institutions in the United States between 2001 and 2014. The crimes covered in the data set are categorized into criminal offenses such as theft, disciplinary actions such as drug violations, hate crimes, VAWA offenses and others.\\
Fortunately (for the sake of safety on campuses), "The Campus Safety and Security" data set is a sparse one (there are many zero values in each record). Therefore, we have to create new versions of data set by combining different records based on our needs. In addition, while the survey is conducted each year for 14 years, a lot of missing data points exist in the datt set which should be taken care of.\\


\noindent
\textbf{Visualization Design:}\\
There are three phenomenons we would like to study and visualize:
\begin{enumerate}
\item The crime statistics for different institutes in each state. Also, see how these statistics have changed over time.
\item Analyze the statistics by choosing a specific type of crime. For each type of crime, the trend for different categories of that crime over time will be shown in graphs.
\item Comparing the crime statistics between different schools. By choosing two or more schools, different types of crime would be compared.
\end{enumerate}
The following are the designs we'd like to consider:\\
\noindent	
\textbf{Crime statistics in each state:} this design would be to display the map of the United States. The map is broken down by states where each one is selectable. By choosing a state, the general crime statistics would be displayed on a line chart in which each line represents a school. Also, when the line in line chart is selected, the detail breakdown of different types of crimes would be displayed on a stack chart. If multiple states are chosen, all graphs are shown on the same chart by which we can do the comparison. Moreover, we have another line chart showing the total number of crimes over time. Finally, we can select one or more schools. When they are selected, their statistics are displayed on a line chart which shows the trend over time.
	
\noindent
\textbf{Comparing different types of crime:} This design also has the map of the United States. The map is the same as described above. Also, there is a rectangle with each corner representing one type of crime (out of four). If a state is selected, a number of circles would be displayed on the rectangle each representing one institution. The location of the circle gives us a sense of the ratio of four different types crime compared to each other. Namely, if the circle leans on the corner representing criminal offense, the proportion of criminal offense in the corresponding school is higher. The area of circle represents the total population of schools in that state. Therefore, this chart could be used to compare different schools. There will be also a slider for year. By changing the presented year, we can observer how the crime statistics have changed over the years. Like the first design, we can also do the selection based on the schools to compare them.
	
\noindent
\textbf{Clustering the data} Our last design will also have the map of the United States. This time we use a scatter plot chart to visualize the data. In this scatter plot, the x-axis represents the total number of crimes while the y-axis represents the number of students. Similar to the second design, circles on the scatter plot represent the schools. Therefore, the ratio of the number of crimes to the number of students could be observed on the chart. The area of circle means the population of the selected selected school. Also, there would be a slider for different year that could be used to observe the how the described ratio has changed over the years.

\noindent
\textbf{Must-Have and Optional Features:}\\
We should be able to easily observe the crime statistics for different schools in each state and be able compare them. We should be able to see how different states are compared to each other in terms of crime statistics in their schools. Also, we should be able to see how these statistics have changed over the years.\\
As mentioned in the objectives, we'd like to combine this data set with other ones containing demographic information, GDP and such for different states and look for correlations between the crime rates and these attributes.

 

%----------------------------------------------------------------------------------------

\end{document}